%%% Artikelklasse mit
%%% * Grundschriftgr��e 11 Punkt,
%%% * klassischem Satzspiegel,
%%% * flachem Inhaltsverzeichnis.
\documentclass[11pt,DIV9,tocleft]{scrartcl}
%%% Eingabekodierung ist ISO-8859-15.
\usepackage[ansinew]{inputenc}
%%% Schrifteinstellung.
%%% * Grundschrift Palatino,
%%% * Akzidenzschrift Bera Sans,
%%% * Schreibmaschinenschrift Luxi Mono.
\usepackage[T1]{fontenc}
\usepackage[osf]{mathpazo}
\usepackage[scaled]{berasans}
\usepackage[scaled=0.85]{luximono}
\linespread{1.02}
\usepackage{textcomp}
%%% Lade einige Pakete.
\usepackage{multicol}
\usepackage{tabularx}
\usepackage{booktabs}
\usepackage{natbib}
\bibliographystyle{plain}
\newcolumntype{L}{>{\raggedright\arraybackslash}X}
%%% Bette Quellkode in PDF-Ausgabe ein.
\usepackage{embedfile}
\embedfile{\jobname.tex}
\usepackage[ngerman]{babel}
%%% Einstellungen f�r interaktive PDF-Dokumente.
\usepackage[rgb,x11names]{xcolor}
\usepackage{hyperref}
\hypersetup{
  pdftitle={Freie Wortlisten und Trennmuster f�r die deutsche Sprache},
  pdfauthor={Die deutschsprachige Trennmustermannschaft}
}
\hypersetup{
  pdfstartview={XYZ null null null},% Zoom factor is determined by viewer.
  colorlinks,
  linkcolor=RoyalBlue3,
  urlcolor=Chocolate4,
  citecolor=SpringGreen3
}
%%% �berschriften in fetter Grundschrift.
\setkomafont{sectioning}{\normalcolor\normalfont\bfseries}
%%% Einige Makros f�r logische Auszeichnungen definieren.
\newcommand*{\Abk}[1]{\textsc{#1}}
\newcommand*{\Programm}[1]{\textsc{#1}}
\newcommand*{\Datei}[1]{\texttt{#1}}
%%% Randnotiz am linken Rand.
\newcommand{\lrand}[1]{%
  \mbox{}%
  \marginpar{%
    \hspace*{-2\marginparsep}%
    \hspace*{-\textwidth}%
    \hspace*{-\marginparwidth}%
    \parbox[t]{\marginparwidth}{%
      \raggedright%
      \small%
      \sffamily%
      \bfseries%
      #1%
    }%
  }%
  \ignorespaces%
}
%%% Satzspiegel erneut berechnen.
\typearea{last}
%%% Trennausnahmen definieren.
\hyphenation{
Back-end
}

\begin{document}
%%% Dokumenttitel.
\author{Die deutschsprachige Trennmustermannschaft}
\title{Freie Wortlisten und Trennmuster\linebreak[1] f�r die deutsche Sprache}
\subtitle{\bigskip Eine kurze Projektbeschreibung}
\setkomafont{title}{\normalcolor\normalfont}
\maketitle
%%% Zweispaltiges Inhaltsverzeichnis.
\begin{multicols}{2}
\tableofcontents
\end{multicols}



\section{Ziele}
Dieses Projekt beabsichtigt hochqualitative Wortlisten,
Trennmuster~\cite{liang:1983} und Ausnahmelisten f�r die deutsche
Sprache zu schaffen.  Teil des Projekts ist eine Infrastruktur, die die
Kontrolle der Listen in verteilter Arbeit erm�glicht.  Wortlisten,
Trennmuster und Ausnahmelisten sollen unter freien Lizenzen jedermann
zug�nglich gemacht werden.



\section{Wer wir sind}
Die deutschsprachige Trennmustermannschaft setzt sich zur Zeit aus
Leuten der Open"-Office-Gemeinde%
\footnote{\url{http://de.openoffice.org/}} und Mitgliedern des
Dante~e.\,V.%
\footnote{Deutschsprachige Anwendervereinigung \TeX~e.\,V.,
  \url{http://www.dante.de/}} zusammen.

Die Kommunikation l�uft derzeit �ber die Gruppe
\Programm{Trennmuster-Opensource} bei Google%
\footnote{\url{http://groups.google.de/group/trennmuster-opensource?hl=de}}%
, das Projekt soll jedoch bei einem anderen Internetdienst zur
Verwaltung von Softwareprojekten angemeldet werden.



\section{Aufgaben}
Dieser Abschnitt enth�lt eine Zusammenstellung von anstehenden Aufgaben.
Die Liste ist weder sortiert, noch vollst�ndig.

\subsection*{Arbeit an Trennmustern}
\begin{itemize}
\item Festlegung "`unserer"' Rechtschreib- und Trennregeln
\item Spezifikation und Gestaltung der Eingabemaske (Frontend)
\item Spezifikation und Implementierung der Datenbankanbindung (Backend)
\item Kontrolle von Rechtschreibung und Trennung des Wortbestands
\item Generierung neuer Trennmuster
\end{itemize}

\subsection*{�nderungen an \TeX}
\begin{itemize}
\item \TeX\ sollte Trennmuster dynamisch laden k�nnen, beispielsweise
  f�r fachspezifische Begriffe ($\Rightarrow$ Lua\TeX).
\item Babel sollte einen Versionsmechanismus f�r die
  Trennmusteraktivierung anbieten, um in einem Dokument trotz
  weiterentwickelter Trennmuster umbruchtreuen Textsatz garantieren zu
  k�nnen.
\item Haupt- und Nebentrennstellen sollten im Trennalgorithmus
  ber�cksichtigt werden.
\end{itemize}

\subsection*{Sonstiges}
\begin{itemize}
\item Umzug auf einen anderen Host (BerliOS, Sourceforge o.\,�.)
\item Reservierung einer eigenen Domain
\end{itemize}
Wer Interesse oder Ideen hat, wende sich bitte an die Gruppe.\newline
\indent\emph{Dieses Projekt ben�tigt Deine Hilfe!}



\section{Ressourcen}
\subsection{Auf welche Wortlisten k�nnen wir zur�ckgreifen?}

\lrand{\medskip Lembergs Liste}
\begin{tabularx}{\linewidth}[t]{lL}
  \toprule\addlinespace
  Urheber & Werner Lemberg\\
  Rechte & \ldots\\
  Wortformen & 420.000\\
  Sortierkriterium & alphabetisch\\
  Rechtschreibung & gut\\
  Bemerkung & manuell gepflegt\\
  Zugriff & \Programm{git}-Repositorium:\newline
  \textbullet\ \verb+git clone git://repo.or.cz/wortliste.git+\newline
  \textbullet\ \verb+git clone http://repo.or.cz/r/wortliste.git+ \\
  Stand & 29.\,3.\,2008\\
  \addlinespace\bottomrule
\end{tabularx}

\bigskip\bigskip
\noindent\lrand{\medskip Leipziger Liste}
\begin{tabularx}{\linewidth}[t]{lL}
  \toprule\addlinespace Urheber & Liste des Wortschatzprojekts der Universit�t Leipzig%
  \footnote{\url{http://wortschatz.uni-leipzig.de/}}\\
  Rechte & GPL\\
  Wortformen & 2.000.000\\
  Sortierkriterium & H�ufigkeit\\
  Rechtschreibung & mangelhaft\\
  Bemerkung & automatische Internetsuche (Datenbanken, Zeitungsarchive usw.)\\
  Zugriff & \ldots\\
  Stand & 28.\,3.\,2008\\
  \addlinespace\bottomrule
\end{tabularx}

\bigskip\bigskip
\noindent\lrand{\medskip Mannheimer Liste}
\begin{tabularx}{\linewidth}[t]{lL}
  \toprule\addlinespace
  Urheber & Korpus des Instituts f�r Deutsche Sprache (\Abk{IDS})%
  \footnote{\url{http://www.ids-mannheim.de/kl/}}\\
  Rechte & ">Darf in ihrer Gesamtheit -- wie vereinbart -- nicht ver�ffentlicht oder an Dritte weitergegeben werden."< Abgeleitete Werke k�nnen nach unserer Wahl behandelt werden.\\
  Wortformen & 4.000.000\\
  Sortierkriterium & H�ufigkeitsklassen\\
  Rechtschreibung & mittel\\
  Zugriff & nicht �ffentlich\\
  Stand & 9.\,10.\,2007\\
  \addlinespace\bottomrule
\end{tabularx}


\subsection{Welche Listen haben wir in Aussicht?}

\lrand{\medskip Berliner Liste}
\begin{tabularx}{\linewidth}[t]{lL}
  \toprule\addlinespace
  Urheber & Kernkorpus des Projekts Digitales W�rterbuch der Deutschen Sprache (\Abk{DWDS})%
  \footnote{\url{http://www.dwdscorpus.de/}}\\
  Rechte & \ldots\\
  Wortformen & 2.000.000\\
  Sortierkriterium & \ldots\\ 
  Rechtschreibung & \ldots\\
  Bemerkung & repr�sentativer Wortschatz der deutschen Sprache\\
  Zugriff & derzeit nicht\\
  Stand & voraussichtlich 2008\\
  \addlinespace\bottomrule
\end{tabularx}



\section{Bisherige Ergebnisse}
\subsection{Wortlisten}
Die von Werner Lemberg erstellte und kontrollierte Liste steht in einem
�ffentlich zug�nglichen \Programm{git}-Repositorium.%
\footnote{\url{http://repo.or.cz/}\qquad Eine \Programm{Git}-Version f�r
  Windows ist unter
  \url{http://code.google.com/p/msysgit/downloads/list} erh�ltlich,
  Datei \Datei{Git-1.5.4-preview20080202.exe} (Stand:
  9.\,2.\,2008).}  Eine Kopie kann mit
\begin{verbatim}
git clone git://repo.or.cz/wortliste.git
\end{verbatim}
oder
\begin{verbatim}
git clone http://repo.or.cz/r/wortliste.git
\end{verbatim}
bezogen werden.%
\footnote{Neben dem Protokoll unterscheiden sich auch die Adressen!}  Im
Repositorium sind auch einige Skripten zur Bearbeitung der Wortliste
enthalten.  Aktualisiert wird die Wortliste (das gesamte lokale
Repositorium) mit
\begin{verbatim}
git pull
\end{verbatim}

\subsection{HTML-Frontend}
Es existiert ein Entwurf f�r eine interaktive Maske zur Kontrolle der
Rechtschreibung von Wortlisten und der Eingabe korrekter Trennungen.%
\footnote{\url{http://www.mnn.ch/opendehyph/index.php}}

\subsection{Trennmusterdateien}
Die bisherigen Trennmusterdateien f�r die traditionelle und neue
Rechtschreibung, die Dateien \Datei{dehypht.tex} und
\Datei{dehyphn.tex}, k�nnen am \Abk{CTAN}%
\footnote{Comprehensive \TeX\ Archive Network,
  \url{http://ctan.tug.org/}} bezogen werden.  Sie sind \emph{nicht} im
Zuge dieses Projekts entstanden.

Im Dateibereich der Google-Gruppe \Programm{Trennmuster-Opensource} sind
Trennmuster verf�gbar, die aus Werner Lembergs Liste generiert wurden.%
\footnote{\url{http://groups.google.de/group/trennmuster-opensource?hl=de}}



\section{Zeitplan}
Es ist kein Ende abzusehen.



\raggedright
\bibliography{projektbeschreibung}


\end{document}

%%% Local Variables: 
%%% mode: latex
%%% TeX-PDF-mode: t
%%% TeX-master: t
%%% End: 
